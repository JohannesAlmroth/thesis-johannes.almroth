\documentclass[11pt,titlepage,openright]{book}
\usepackage[utf8]{inputenc}
\usepackage[T1]{fontenc}
\usepackage[british]{babel}
\usepackage{graphicx}
\usepackage[dvipsnames]{xcolor}
\usepackage{libertine}
\renewcommand*\ttdefault{cmtt}

\usepackage[sf]{titlesec}
\usepackage[square,sort,colon,authoryear]{natbib}
%\usepackage[export]{adjustbox}

\usepackage{
  paralist,   % Improved lists 
  marginnote, % Improved margin notes
  environ,
  ragged2e,   % Justified text in the margin notes
  url,        % For typesetting URLs
  listings,   % Code formatting
  hyperref,   % Links in PDF from TOC, refs, etc.
  lipsum
}
\usepackage{changepage}
\usepackage{float}

\usepackage[twoside,labelfont=sf]{caption}

\captionsetup{justification=raggedright,singlelinecheck=false}

\newcommand\myhrulefill[1]{\leavevmode\leaders\hrule height#1\hfill\kern0pt}
\DeclareCaptionFormat{FigFormat}{{\color{black}\myhrulefill{0.5pt}}\\#1#2#3}
\captionsetup[figure]{format=FigFormat}
\captionsetup[table]{format=FigFormat}

\DeclareCaptionFormat{LstFormat}{\textsf{Listing}~\arabic{chapter}.\arabic{listing}:#2#3}
\floatstyle{ruled}
\newfloat{listing}{thp}{lol}
\floatname{listing}{Listing}
\captionsetup[listing]{format=LstFormat}


\NewEnviron{MarginNote}[1][0mm]{\marginnote{\footnotesize\justifying\BODY}[#1]}
\newcommand{\Footnote}[2][0mm]{\footnotemark\marginnote{\footnotesize$^{\arabic{footnote})}$~#2}[#1]}

\renewenvironment{figure*}[1][]{%
  \begin{figure}[#1]%
    \checkoddpage%
    \ifoddpage%
      \begin{adjustwidth}{0cm}{-45mm}%%
    \else%
      \begin{adjustwidth}{-45mm}{0cm}%%
    \fi%
    }{%
    \end{adjustwidth}%
  \end{figure}}

\renewenvironment{table*}[1][]{%
  \begin{table}[#1]%
    \checkoddpage%
    \ifoddpage%
      \begin{adjustwidth}{0cm}{-45mm}%%
    \else%
      \begin{adjustwidth}{-45mm}{0cm}%%
    \fi%
    }{%
    \end{adjustwidth}%
  \end{table}}

\renewenvironment{listing*}[1][]{%
  \begin{listing}[#1]%
    \checkoddpage%
    \ifoddpage%
      \begin{adjustwidth}{0cm}{-45mm}%%
    \else%
      \begin{adjustwidth}{-45mm}{0cm}%%
    \fi%
    }{%
    \end{adjustwidth}%
  \end{listing}}

%% == Code =======================================================
\lstnewenvironment{Code}[1][style=std]{\lstset{#1}}{}
\lstnewenvironment{Code_Numbered}[1][style=std,numbers=left]{\lstset{#1}}{}

\renewcommand{\c}[1]{\lstinline[style=std]@#1@}

\lstdefinestyle{std}{
  language=java,
  basicstyle=\small\sf\color{black},
  keywordstyle=\small\sf\bfseries,
  numberstyle=\footnotesize\sf\color{black},
  commentstyle=\small\color{black}\it,
  aboveskip=1ex,
  belowskip=1ex,
  tabsize=2,
  columns=fullflexible,
  xleftmargin=1ex,
  resetmargins=true,
  showstringspaces=false,
  morecomment=[l]{//},
  morecomment=[l]{--},
  morecomment=[s]{/*}{*/},
  escapeinside=@@,
  morekeywords={Frobies},
  moredelim=[is][\textit]{___}{___},
  moredelim=[is][\textbf]{__*}{*__}
}

\usepackage[activate={true,nocompatibility},final,tracking=true,kerning=true,spacing=true,factor=1100,stretch=10,shrink=10]{microtype}
\usepackage[paper=a4paper,text={13cm,24cm},marginparsep=5mm,marginparwidth=45mm,inner=20mm,twoside]{geometry}

\newcommand{\RED}[1]{\textcolor{red}{#1}}
\newcommand{\ie}{\emph{i.e.,}}
\newcommand{\eg}{\emph{e.g.,}}
\newcommand{\etal}{\emph{et~al.}}

\renewcommand{\bfdefault}{b}
\clearpage{\pagestyle{empty}\cleardoublepage}

\synctex=1
\pagestyle{plain}

\begin{document}
\frontmatter
\title{Thesis Template}
\author{Tobias Wrigstad}
\date{\today}

\maketitle

\vspace*{3cm}
\section*{Abstract}

\RED{Replace this with the actual abstract. Obviously.}
Suspendisse luctus leo et porta mattis. In semper, nisi et
suscipit iaculis, leo urna laoreet lacus, ut laoreet lorem tellus
eget dui. Vestibulum eu auctor nisi. Morbi pharetra euismod velit
ac mattis. Maecenas tempor vitae augue ut aliquam. Nunc auctor,
nibh at imperdiet finibus, ex leo semper lacus, ac vehicula quam
nisl condimentum leo.


\tableofcontents
\listoffigures
\listoftables

\mainmatter

\chapter{Introduction}

\lipsum[1]

\begin{MarginNote}[3cm]
  \lipsum[2]

  \RED{Use linebreaks\-to\-insert\-hyphens\-where\-needed\-in the margin.}
\end{MarginNote}

Footnotes\Footnote[-1cm]{Like this one. Note the manual vertical offset in the source code.} will go into the margin. They should be used sparingly.

\lipsum[3]

\section{Purpose and Goals}

\lipsum[4-6]

\section{Thesis Outline}

\lipsum[7]

\begin{figure}[t]
  \begin{center}
    \vspace*{1em}
    \verb+\includegraphics[width=0.9\linewidth]{image.pdf}+
    \vspace*{1em}
  \end{center}
  \caption{This is a figure, it will be placed top-most of a page (if possible).}
  \label{fig:example-1}
\end{figure}

\begin{figure*}[t]
  \begin{center}
    \vspace*{1em}
    \verb+\includegraphics[width=0.9\linewidth]{image.pdf}+
    \vspace*{1em}
  \end{center}
\caption{This is a wider figure, it too will be placed top-most of a page (if possible).}
  \label{fig:example-1}
\end{figure*}


\begin{table}[b]
  \caption{This is a table, it will be placed at the bottom of a
    page (if possible). Note that captions for tables are placed
    \emph{above} the table.}
  \begin{center}
    \begin{tabular}{|l|l|l|}
      \hline
      $A$ & $B$ & $C$ \\
      \hline
      $A$ & \multicolumn{2}{|c|}{$B$ \& $C$} \\
      \hline
      $A$ & $B$ & $C$ \\
      \hline
    \end{tabular}
  \end{center}
  
  \label{tab:example-1}
\end{table}

\chapter{Background}

This chapter typically contains a discussion of the related work.
BibTeX is a great way to manage bibliography information. If you
use Emacs, tools like RefTeX will help you insert references into
your code. Given that we are not constrained for space, we can use
citations that look like this: \citep{Acm_Curriculum_2013,
  CWE_193}. To give the text a bit better flow, you can sometimes
work them into the text. Using numbered references is allowed but
it will cause them to be much like Dijkstra's
\citeyearpar{Dijkstra_1968} dreaded GOTO's.

\chapter{Methodology}
\lipsum[1] Code \c{int inline = ++x;} looks like so:

\begin{Code}
public class HelloWorld<T> extends Something<T> {
  public static void main(String[] args) {
    System.out.println("Hello, world!");
  }
}
\end{Code}

You can also get it with line numbers:

\begin{Code_Numbered}
public class HelloWorld<T> extends Something<T> {
  public static void main(String[] args) {
    System.out.println("Hello, world!"); @\label{code:hello}@
  }
}
\end{Code_Numbered}


You can refer to lines like this: \verb+Line \ref{code:hello}+ (gives you Line \ref{code:hello}, see source for more details).

Verbatim text is set in a monospaced typewriter font:

\begin{verbatim}
----------,
          |           ASCII art!
          '--> (*)
\end{verbatim}

\lipsum[2-4]

\begin{listing*}[t]
\begin{Code_Numbered}
public class HelloWorld<T> extends Something<T> {
  public static void main(String[] args) {
    System.out.println("Hello, world!"); @\label{code:hello}@
  }
}
\end{Code_Numbered}
\caption{If you like, you can have code listings inside a \c{listing} float.}
\end{listing*}



\chapter{Design}
\lipsum

\chapter{Implementation}
\lipsum

\chapter{Evaluation}
\lipsum

\chapter{Conclusions}
\lipsum

\bibliographystyle{plainnat}
\bibliography{main}

\end{document}

%%% Local Variables: ***
%%% mode: latex ***
%%% TeX-master: "main.tex"  ***
%%% ispell-local-dictionary: "british"  ***
%%% End: ***