IoT technology has been proclaimed as a new technological prowess that will change our economy as well as our cities and way of living. Despite these bold statements, IoT is far from being easily implemented by companies not directly working with any of the enabling technologies, such as telecom. Narrowband-IoT (NB-IoT), a new radio protocol focusing on wide area coverage and low power consumption, is being heralded by the 3GPP as one of the key technologies necessary to push society into the age of IoT. NB-IoT networks are still extremely new in a lot of countries, and while the SIM-cards necessary to use these networks can be readily purchased from telecom companies, the lack of implemented projects might scare the everyday layman looking to implementing IoT within his/her business. The purpose of this paper is to provide an example for how an IoT device can be implemented in practicality, specifically with a scale. A micro-controller is hooked up to a load cell, from which the data produced is sent to the net via a cloud platform. 