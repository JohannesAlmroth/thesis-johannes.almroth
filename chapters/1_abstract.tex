IoT technology has been proclaimed as a new technological prowess that will change our economy, our cities and our way of living. Despite these bold statements, IoT is far from being implemented by
ordinary tech companies not directly working with any of the enabling technologies, such as telecom. For IoT to live up to the excitement and predictions surrounding it, the hardware and the technical know-how needs to be accessible in the form of reasonable price and complexity as well as increased availability in the form of infrastructure and networks. 
Narrowband-IoT, a new radio protocol focusing on wide area coverage and low power consumption, is being heralded by the 3GPP as one of the key technologies necessary to push society into the age of IoT. Narrowband-IoT networks are still extremely new in a lot of countries, and while the SIM-cards necessary to use these networks can be readily purchased from telecom companies, the lack of implemented projects might scare the companies looking to implementing IoT within their business. The purpose of this paper is to provide an example of how an IoT device can be implemented in practicality, with the focus being on using a load cell as a sensor. 

Using a microcontroller, a hardware connection with a load cell is implemented. Due to time constraint, the data transmission tests are done with a virtual dataset using Wi-fi. 