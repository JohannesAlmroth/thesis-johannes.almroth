\iffalse
\begin{itemize}
	\item Identifier names are important
	\item We have two identifier styles
	\item Some history 
	\item Two other studies
	\item Some background on them
	\item Reference all terms introduced
\end{itemize}
\fi

Naming variable identifiers in a effective and efficient way is important for code comprehension. Good identifiers enables, for example, abstraction of concepts, collaboration and code preservation. This is put into perspective when reading a paper by Deissenboeck \etal \cite{Concise_Naming} which states that 70\% of the source code for the popular Java IDE program \textit{Eclipse} consists of identifiers. The two main styles of naming identifiers in code today are camelCasing (\eg coolBeans) and underscore (\eg cool\_beans). Deciding which one to use is often a matter of the given convention within the programming language that the code is written in. So far there's been two other studies aiming to determine if there is a significant difference between the two, and if there is, which one is better. The first study conducted by Binkley \etal 


\section{Purpose and Goals}
\iffalse
\begin{itemize}
	\item Write about the grand scheme of things
	\item Set the correct expectations
	\item What can I expect to learn if I keep on reading?
	\item What are the success criteria for this work?
	\item How will the work be evaluated?
\end{itemize}
\fi

\section{Delimitations}
\iffalse
\begin{itemize}
	\item Scale down expectations and clarify
\end{itemize}
\fi

I will limit the project by not extending the original study with more than a few additional variables that are relevant to the subject. I will furthermore limit the size of participants to the minimum amount necessary to achieve statistical significancy, if time doesn't permit otherwise. All data gathered will solely be through the same means used in the original 2010 study, the tobii eye tracker as well as interviews.