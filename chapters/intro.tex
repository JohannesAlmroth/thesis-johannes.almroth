* Identifier names are important

* We have two identifier styles

* Some history 

* Two other studies

* Some background on them

* 

\begin{MarginNote}[3cm]
  \lipsum[2]

  \RED{Use linebreaks\-to\-insert\-hyphens\-where\-needed\-in the margin.}
\end{MarginNote}

Footnotes\Footnote[-1cm]{Like this one. Note the manual vertical offset in the source code.} will go into the margin. They should be used sparingly.


\section{Purpose and Goals}

\lipsum[4-6]

\section{Thesis Outline}

\lipsum[7]

\begin{figure}[t]
  \begin{center}
    \vspace*{1em}
    \verb+\includegraphics[width=0.9\linewidth]{image.pdf}+
    \vspace*{1em}
  \end{center}
  \caption{This is a figure, it will be placed top-most of a page (if possible).}
  \label{fig:example-1}
\end{figure}

\begin{figure*}[t]
  \begin{center}
    \vspace*{1em}
    \verb+\includegraphics[width=0.9\linewidth]{image.pdf}+
    \vspace*{1em}
  \end{center}
\caption{This is a wider figure, it too will be placed top-most of a page (if possible).}
  \label{fig:example-1}
\end{figure*}


\begin{table}[b]
  \caption{This is a table, it will be placed at the bottom of a
    page (if possible). Note that captions for tables are placed
    \emph{above} the table.}
  \begin{center}
    \begin{tabular}{|l|l|l|}
      \hline
      $A$ & $B$ & $C$ \\
      \hline
      $A$ & \multicolumn{2}{|c|}{$B$ \& $C$} \\
      \hline
      $A$ & $B$ & $C$ \\
      \hline
    \end{tabular}
  \end{center}
  
  \label{tab:example-1}
\end{table}