\iffalse
\begin{itemize}
	\item Identifier names are important
	\item We have two identifier styles
	\item Some history 
	\item Two other studies
	\item Some background on them
	\item Reference all terms introduced
\end{itemize}
\fi

Naming variable identifiers in a effective and efficient way is important for code comprehension. Good identifiers enables, for example, abstraction of concepts, collaboration and code preservation. This is put into perspective when reading a paper by Deissenboeck \etal \cite{Concise-Naming} which states that 70\% of the source code for the popular Java IDE program \textit{Eclipse} consists of identifiers. The two main styles of naming identifiers in code today are camelCasing (\eg coolBeans) and underscore casing (\eg cool\_beans). \footnote{Underscore casing is more commonly known as snake casing, though the previous expression will be used to be more consistent with the established literature}

Deciding which one to use is often a matter of the given convention within the programming language that the code is written in. Historically, the arguments for using camel casing is that it requires fewer keystrokes and improves typing speed. 

So far there's been two other studies aiming to determine if there is a significant difference between the two styles. The first study conducted by Binkley \etal in 2009 \cite{to-camelCase-or-under-score} with 135 subjects concluded that the camel casing style leads to a better all round performance, at least when the subject is trained on the style, despite taking on average 0.42 seconds longer to read. The second study conducted by Sharif \etal in 2010 \cite{eye-tracking-study} with the help of eye-tracking equipment found that camel cased words took on average 0.932 seconds longer to read, and concludes that the under score style leads to an improvement in both reading time and visual effort. Even though the two studies differ in their conclusions, the eye-tracking study suffers from a small sample size, with a meagre 15 subjects compared to the 135 subjects in the 2009 study. However, the eye-tracking equipment used in the 2010 study lends a lot of credibility to the data presented.

This study will also be conducted with the help of eye tracking equipment and will aim to provide additional data that would further aid in the judgement regarding which one of these two styles would be the most beneficiary for code comprehension. 


\section{Purpose and Goals}
\iffalse
\begin{itemize}
	\item Write about the grand scheme of things
	\item Set the correct expectations
	\item What can I expect to learn if I keep on reading?
	\item What are the success criteria for this work?
	\item How will the work be evaluated?
\end{itemize}
\fi

\section{Delimitations}
\iffalse
\begin{itemize}
	\item Scale down expectations and clarify
\end{itemize}
\fi

I will limit the project by not extending the original study with more than a few additional variables that are relevant to the subject. I will furthermore limit the size of participants to the minimum amount necessary to achieve statistical significancy, if time doesn't permit otherwise. All data gathered will solely be through the same means used in the original 2010 study, the Tobii eye tracker as well as interviews.