\iffalse
\begin{itemize}
	\item Identifier names are important
	\item We have two identifier styles
	\item Some history 
	\item Two other studies
	\item Some background on them
	\item Reference all terms introduced
\end{itemize}
\fi

Naming variable identifiers in a effective and efficient way is important for code comprehension. A programmer can only get so far with using variables like \textit{x} and \textit{y}, and to write code that encapsulates complex mental models of thinking that need to be understood by multiple people in a development team, appropriate identifier names need to be chosen. Good identifiers enables, for example, abstraction of concepts, collaboration and code preservation. This is put into perspective when reading a paper by Deissenboeck \etal \cite{Concise-Naming} which states that 70\% of the source code for a given release of the popular Java IDE program \textit{Eclipse} consisted solely of identifiers. But assuming that the wording of the identifiers is sufficient, the question remains of \textit{how} they can be written to further aid in code comprehension.

The two main styles of writing identifiers today are camelCasing (\eg coolBeans) and underscore casing (\eg cool\_beans). \footnote{Underscore casing is more commonly known as snake casing, though the previous expression will be used to be more consistent with the established literature}

Deciding which one to use is often a matter of the given convention within the programming language that the code is written in. As noted in a previous study on the subject \cite{eye-tracking-study}, early programming languages such as Basic, Cobol, Fortran and Ada were case-insensitive, and thus encouraged the use of either the underscore or hyphens to write variables. When languages such as C and Java were introduced, camel casing became more common, and the argument can be made that it requires fewer keystrokes, and thus improves writing speed. However, natural language research suggests that this is the wrong approach. A paper by Epelboim \etal \cite{Fillers-in-text} found that un-spaced text lead to a 10-20\% slower reading speed in subjects, while characters similar to the underscore had the less impact on reading speed overall. So while there is a short term gain for the programmer in the form of typing speed, it could lead to a steeper cost in code maintainability. How severe this affects maintainability is outside the scope of this paper.

As of today there's been two studies aimed at answering the question which one of camel casing or underscore casing affects reading comprehension the most. These are papers by Binkley \etal \cite{to-camelCase-or-under-score} and Bonita \etal \cite{eye-tracking-study} conducted in 2009 and 2010 respectively. The latter aimed to replicate the former, and they both draw different conclusions from their findings.





\section{Purpose and Goals}
\iffalse
\begin{itemize}
	\item Write about the grand scheme of things
	\item Set the correct expectations
	\item What can I expect to learn if I keep on reading?
	\item What are the success criteria for this work?
	\item How will the work be evaluated?
\end{itemize}
\fi

This study will aim to provide additional data that would further aid in the judgement regarding which one of these two styles would be the most beneficiary for code comprehension. Furthermore, it will use data gathering equipment in the form of an eye-tracker, and aim to achieve a solid statistical significance through a high enough subject count.

\section{Delimitations}
\iffalse
\begin{itemize}
	\item Scale down expectations and clarify
\end{itemize}
\fi

The study will be limited by primarily aiming to replicate the 2010 study. The same test data and variables will be gathered, as well as focusing on acquiring more subjects. Additionally, a few extra variables about the subjects will be gathered pertaining to previous programming experiences as well as their identifier style of choice, as to expand the given research and to contribute data to the [...] (research area?)