% Should be ~2 pages 

This section aims to give the reader a bit more background into the circumstances of the technologies used in the paper. We will discuss the background of the NB-IoT protocol and how it affected the groundwork of this paper, as well as expand on some of the concepts mentioned in the previous chapter. This is relevant information for people seeking to replicate the paper in some part, whether it's about acquiring and working with similar hardware or making basic software design choices about energy efficacy. 

\textbf{TODO: Is the Vetek paragraph even worth mentioning?}Vetek is a Swedish scale supplier located in Väddö, situated approx. 100 kilometers north of Stockholm. Vetek constructs their own scales and weighing systems, as well as reselling products from other manufacturers.\cite{vetek} Vetek aims to improve their services, and as such are interested in the possible use cases of IoT (Internet of Things) technology, and ultimately see how that can be applied to their own products. In a pilot project, Vetek wants to see how this connectivity can be implemented in a energy-efficient and effective manner. This would entail investigating factors such as power consumption, range and data rate.


As mentioned in the previous chapter, there are pretty high hopes regarding the expansion and profitability of the IoT industry as a whole. To enable this growth and functionality, the 3GPP developed various new radio technologies, the two most prominent being NB-IoT and LTE-M. LTE-M has the most functionality, including voice capabilities and device positioning. Thanks to its wider bandwidth frequency it also has lower latency and boasts a data rate up to 1 Mbps.\cite{ericsson-blog} In return, device complexity and costs are higher compared to NB-IoT. The focus of NB-IoT was to enable indoor coverage, low cost development, long battery life and high connection density, which makes the technology ideally suited for low data rate applications in extremely challenging radio conditions. 

Because of the novelty of IoT and the NB-IoT, the technology isn't easily accessible, and the required SIM-cards are only available to purchase for a hefty sum. As of writing this paper, a trial-kit from Telia (in Sweden), containing 5 SIM-cards for a period of 6 months currently costs €450.\cite{telia-nb}  Compared to other communication protocols, they are not as simple to implement out of the box, but the potential benefits should be enticing enough for many innovators to start experimenting.

According to Swedish telecom company Telia, they were the first to introduce the NB-IoT technology in Sweden, as well as the Nordic countries overall.\cite{telia-nb} They further claim that their network will be in range for over 99.9\% of Sweden's population, as well as provide a speed of 200 kb/s in more than 95\% of the country.\cite{telia-first} The grand opening of the network was on the 24th of May, and pilot projects were conducted as early as a year before this, in multiple locations across the country. Telia currently offers a starter kit  for any actor interested in the technology, with a trial period of 6 months that includes access to Telia's IoT portal and APIs as well as 5 SIM cards, each with a 30MB data cap per month. Telia doesn't seem have many competitors when it comes to the Swedish IoT market, though Tele2 have partnered up with Nokia to offer similar services, and according to a press release from 2018, they have rolled out both LTE-M and NB-IoT across their networks.\cite{tele2-nokia} Telenor has launched a IoT network in Norway with NB-IoT functionality in 2018\cite{telenor-iot}, and according to an exchange with their customer support, followed suite in Sweden in the beginning of October. \textbf{TODO: How do I reference a private email conversation?} The fact that Telia already has partnered up with a multitude of cities and companies give the indication that they have a head start in the market.


\section{Requirements}
Aside from security, one of the biggest challenges regarding IoT devices relate to limitations arising from energy infrastructure. As mentioned earlier, one of the core issues NB-IoT aims to achieve is to be a low-power technology, thus decreasing the maintenance needed for battery-powered devices. A claim often paraded with NB-IoT is that it enables a battery-time of up to 10-years\cite{gsma-nb-iot}, though it's worth mentioning that over such a period of time the underlying IoT technology (in the form of microcontrollers/sensors) will probably require more frequent maintenance than the batteries themselves. However, if an IoT device constantly transmitted data for days on end, its energy supply would run out rather quickly. Therefore, it's also important that energy consumption is something that's accounted for when writing the code for an IoT application. Questions worth considering are \textbf{how often} and \textbf{when} to send data, in order to ensure device effectiveness while still maintaining energy efficiency. 

After questioning Vetek about possible use cases that a battery-powered IoT device would fit into, the most common examples boiled down into monitoring weights that would change in a linearly decreasing fashion. 

\textbf{TODO: Explain why the three following subsections were chosen}

\subsection{Data Reading Interval}
In most IoT devices the relevant data is provided by some form of sensor, whether it be a scale, thermometer or something entirely different. The type of sensor being used has a huge impact on the IoT device, especially when considering that they have to be powered by the same energy source. However, it's safe to assume that in most cases (depending on the sensor), the data transmissions will be the part of the device that will consume the most amount of energy during the lifetime of the device. Nonetheless, it's also important to factor in how often the device polls the sensor for data. The simplest way of deciding when to poll data from a sensor is to let it do so at a fixed and constant rate, often enough to be relevant, and seldom enough as to not waste precious energy. The same can be said for the actual transmission of the data, though this will be discussed elsewhere in the paper \textbf{TODO: Where?}

However, if within the context of the application we can conclude that no data needs to be polled (for a while), then subsequently no data will need to be sent, and thus we save energy on both ends of the system. For some applications there might even be longer periods of downtime where it's not relevant to conduct monitoring on the given sensor, \eg during nighttime, closing hours, etc. Another interesting angle is modifying the reading rate depending on the data itself. A simple example of this would be to have a slower reading interval at stable values, and increase it when experiencing large enough changes. Given the conditions of an IoT device powered by batteries, it's not unreasonable to assume that readings might not always be accurate at times. Depending on the sensor, spikes and drops of false values might occur, and not taking these scenarios into account would be prudent. In the following chapter we will explore a possible implementation regarding the reading rate of sensor data depending on the output of the data values.

\subsection{Sensor Failure}

In this paper, we define sensor failure as a sensor giving too many unreliable or false data values to be considered functional. The goal of identifying such a state in an IoT device is to prevent unstable data from being interpreted as valid, which in turn can save the end user from unwanted consequences. Depending on the longevity and purpose of the device, the threshold of when to declare a sensor as failing may differ, especially as this state can be quite fluid. A functional sensor means different things for different devices and applications. A simple way might be to conclude that if x\% of data is considered invalid during the last 24hrs, an alarm should be raised to the device administrator. Complications arise when failures need to be reported quickly, or estimated more thoroughly. It's also possible that the sensor can be temporarily unreliable due to external circumstances, and given enough time, these circumstances might \textbf{TODO: vanish/recede/pass}. On one extreme you can have a device that reports failures too frequently and bogs down whatever dashboard is handling it's status report. On the other, you can have a device taking too long to determine a sensor failure that otherwise useful data could have been monitored if an error had been raised in time. In the following chapter we will look at possible way to handle sensor failure in a somewhat fluid manner, with the goal of being responsive while still allowing the sensor some leeway.




\subsection{Sensor Disconnect}
We define a sensor disconnect as when no credible data is being produced at all. If the sensor doesn't recover, immediate maintenance is needed for any continued functionality. If 