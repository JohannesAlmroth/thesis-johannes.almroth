\iffalse
\begin{itemize}
	\item How to measure simplicity and scalability
	\item Not just what you did but why and what the consequences are
\end{itemize}
\fi

\section{Research questions}
Explore design space, motivate method, consequences, 


I will attempt to replicate the 2010 study using an eye tracker device from tobii. I will set up a test process as similar to the original as I can, as to gather a minimum of the same amount of data. My aim is also to gather more data about the subjects that could act as confounding variables, such as what programming languages that they've used historically and currently, age gap, and other variables TBD (to be determined).

In the original test process subjects were instructed to observe an identifier on a screen, which was then replaced with another four identifiers. One of them matched the previous identifier, whereas the rest of them were incorrect, meant ot distract the subject. The data that was then measured is the accuracy of the subject to identify the correct word, as well as the time it took to reach an answer. From the tobii eye tracker additional data such as time spent on each word, amount of fixations vs saccades (eye flicker), can be extracted.
Furthermore, data such as subject age, experience and background will be extracted in an interview like setting.

\section{Eye-tracking Equipment}

\section{Material and Stimuli}

\section{Visual Effort and Areas of Interest}

\section{Study Variables}

\section{Hypotheses}

\section{Participants}

\section{Instrumentation}