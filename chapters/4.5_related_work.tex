% TODO: Rewrite?
A similar project done at KTH in 2019 served as the main inspiration for this paper.\cite{hospital} The goal of the project was to monitor the battery levels in defibrillators via an IoT-enabled scale, and thereby optimize the battery-swapping routine. This project proved successful in it's implementation, but faced a slew of challenges that prevented it from ever being integrated into an actual user environment, due to security concerns related to the use of the hospital Wi-Fi. This shows the importance of seeing the bigger picture of where and how the device will be implemented by the end-user, and having that in mind when making technology and design trade-offs throughout the development process.

% Something related to data polling
% TODO: Cite ADP-MAC that adjusts itself on incoming traffic
One of the implementation details in this project concerned the data polling rate, which could potentially affect battery life of a sensor. The ways to regulate data polling is as varied as there are systems and devices that implement it, and apart from looking at the the energy consumption between the polling unit and the sensor, the following works look at the aspect of data polling in the context of a system, which is particularly interesting in the perspective of IoT since multiple devices will often share the same network and need to optimize the use of shared resources.

In a 2018 study by Siddiqui \etal \cite{ADP-MAC} a new protocol was developed for the MAC layer of a wireless sensor network which adjusted its polling interval depending on the rate of incoming traffic, and given certain types of traffic, was able to optimize energy and delay performance compared to another MAC protocol. 
% TODO: Cite nanonetwork polling that adjusts itself to network conditions
In a work by Yu \etal concerning electromagnetic-based wireless nano sensor networks, a polling scheme is proposed that adjusts itself according to network conditions on the IoT backhaul portion of the system, which improved bandwidth efficiency and lessened energy consumption. 

% Something related to error detection.
% TODO: Find at least two articles relating to this