\iffalse
\begin{itemize}
  \item Have a specific background or spread it out
  \item May be necessary to introduce certain things
  \item From our running example, we might discuss prior work on simplifying parallel programming, or SOTA with e.g., threads
\end{itemize}
\fi

In 2010 a study published at ICPC examined how identifier naming conventions affect code comprehension with the help of eye tracking equipment. The two styles that were analysed were camelCase as well as under\_score identifiers. The goal of this project is to try and replicate the results of the study with the same method, as well as taking more variables into account, if time and the sample size of the study group permits.

Identifier names are keystones in software programming for presenting and working with data in any kind of function. If any way of writing these identifiers improves the speed of which we comprehend the code, it could potentially entail better overall program understanding.

Historically, underscores were used as the primary identifying style because of early programming languages being case insensitive. Later on, camel-case identifiers became the norm, maybe partially because of the number of keystrokes and ease of writing it had over using underscores.

The results from the 2010 study concluded that there were no significant differences in regards to accuracy between the two styles. However, the camel-cased identifiers took longer to identify. The 2010 study in turn sought to replicate a study conducted in 2009, which similarly concluded that camel-cased identifiers were slower to identify, but that underscored identifiers brought about less accuracy. It's worth mentioning that the two studies differed their methods of gathering data, as the 2009 study used a game like interface to gather timed responses, while the 2010 study used eye tracking equipment to gather more quantitative data.
