This chapter typically contains a discussion of the related work.
BibTeX is a great way to manage bibliography information. If you
use Emacs, tools like RefTeX will help you insert references into
your code. Given that we are not constrained for space, we can use
citations that look like this: \citep{Acm_Curriculum_2013,
  CWE_193}. To give the text a bit better flow, you can sometimes
work them into the text. Using numbered references is allowed but
it will cause them to be much like Dijkstra's
\citeyearpar{Dijkstra_1968} dreaded GOTO's.

\chapter{Methodology}
\lipsum[1] Code \c{int inline = ++x;} looks like so:

\begin{Code}
public class HelloWorld<T> extends Something<T> {
  public static void main(String[] args) {
    System.out.println("Hello, world!");
  }
}
\end{Code}

You can also get it with line numbers:

\begin{Code_Numbered}
public class HelloWorld<T> extends Something<T> {
  public static void main(String[] args) {
    System.out.println("Hello, world!"); @\label{code:hello}@
  }
}
\end{Code_Numbered}


You can refer to lines like this: \verb+Line \ref{code:hello}+ (gives you Line \ref{code:hello}, see source for more details).

Verbatim text is set in a monospaced typewriter font:

\begin{verbatim}
----------,
          |           ASCII art!
          '--> (*)
\end{verbatim}

\lipsum[2-4]

\begin{listing*}[t]
\begin{Code_Numbered}
public class HelloWorld<T> extends Something<T> {
  public static void main(String[] args) {
    System.out.println("Hello, world!"); @\label{code:hello}@
  }
}
\end{Code_Numbered}
\caption{If you like, you can have code listings inside a \c{listing} float.}
\end{listing*}
