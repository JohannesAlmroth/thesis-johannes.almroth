\iffalse
\begin{itemize}
	\item Identifier names are important
	\item We have two identifier styles
	\item Some history 
	\item Two other studies
	\item Some background on them
	\item Reference all terms introduced
\end{itemize}
\fi

Naming identifiers in a effective and efficient way is important for code comprehension. Variables such as \textit{x} and \textit{y} are not enough to convey context as well as meaning, and to write code that encapsulates complex mental models of thinking that need to be understood by multiple people in a development team, appropriate identifier names need to be chosen\cite{naming-and-code-quality, code-quality-metric, identifier-study}. Good identifiers increases code readability, and that in turn is critical for the abstraction of concepts, collaboration and code maintainability\cite{readability-maintenance, identifier-study}. This is put into perspective when reading a paper by Deissenboeck \etal{} \cite{Concise-Naming} which states that 70\% of the source code for a given release of the popular Java IDE program \textit{Eclipse} consisted solely of identifiers (measured in characters). Given the importance of identifiers, the question remains of \textit{how} they can be written to further aid in code comprehension.

The two main styles of writing identifiers today are camelCasing (\eg{} coolBeans) and underscore casing (\eg{} cool\_beans). Deciding which one to use is often a matter of the given convention within the programming language that the code is written in. As noted in a previous study on the subject \cite{eye-tracking-study}, early programming languages such as Basic, Cobol, Fortran and Ada were case-insensitive, and thus encouraged the use of either the underscore or hyphens to write variables. When languages such as C and Java were introduced, camel casing became more common, and the argument can be made that it requires less effort to type, and thus improves writing speed. A paper by Epelboim \etal{} \cite{Fillers-in-text} found that un-spaced text (\ie{} hellothere) lead to a 10--20\% slower reading speed in subjects, and would thus suggest that camel cased identifiers should be harder to read than snake cased identifiers.

As of today there's been two studies aimed at answering the question which one of camel casing or underscore casing affects reading comprehension the most. These are papers by Binkley \etal{} \cite{to-camelCase-or-under-score} and Bonita \etal{} \cite{eye-tracking-study} conducted in 2009 and 2010 respectively. The findings of these studies will be presented in [...]





\section{Purpose and Goals}
\iffalse
\begin{itemize}
	\item Write about the grand scheme of things
	\item Set the correct expectations
	\item What can I expect to learn if I keep on reading?
	\item What are the success criteria for this work?
	\item How will the work be evaluated?
\end{itemize}
\fi

The study will primarily replicate the 2010 paper. This study aims to provide additional data to the questions posed in the 2009 and 2010 papers, \eg{} how does camelCasing and snake\_casing affect code comprehension. The data gathering equipment used will be in the form of an eye-tracker. Another goal of the study is to achieve a solid statistical significance through a high enough subject count. Additionally, a few extra variables about the subjects will be gathered pertaining to previous programming experiences as well as their identifier style of choice, as to expand the given research and to contribute additional data to the [...] (research area?)


\section{Delimitations}
\iffalse
\begin{itemize}
	\item Scale down expectations and clarify
\end{itemize}
\fi
This study will only use the same test format as presented in the 2010 paper. 