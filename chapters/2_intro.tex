% % Should be 25% of the total paper, i.e. 3-5 pages
% **Problems**

% - Too much detail (technical and about industry)
% - Technically detailed stuff, both load cell and LTE-M vs NB-IoT


% **Needs to contain**

% - Example of real world applications
% - High level concepts only regarding technical implementation
% - Figures & drawings

% **Structure**

% - IoT general info
% - Intro about Vetek
% - Vetek + IoT
% - Challenges
% - Available tech
% - What I did
% - Road map to other chapters

% Paragraph of IoT
Internet of Things (IoT) is a broad, diverse and growing field within the IT sector. Many organizations predict that it will come to impact large areas of our daily life, and many telecom companies are experimenting with different kinds of real world applications that can benefit from this change. The basic idea is the same for all devices, which is to communicate via the internet and/or with other device nodes in a network, without the supervision or interaction of a human. Examples range from simple toasters to complex self-driving cars \cite{what_is_iot}. The focus is to enable communication between devices without the need for a human middleman, thus optimizing whatever application is being implemented. A simple example is a building equipped with multiple IoT-enabled thermostats, which are controlled by a central heating system. Given effective software, heating can be regulated in an energy efficient manner while still keeping visitors adequately warm throughout the day. Another example might be a parking meter, which can forward the availability of its parking spot to some central system which in turn forwards the closest available spot to an end-user. The potential applications are numerous, but factors such as energy consumption and security have proven to be roadblocks that pose considerable challenges to most IoT projects.

% Intro about Vetek
Vetek is a Swedish scale supplier located near Väddö island, situated approx. 100 kilometers north of Stockholm. Vetek constructs their own scales and weighing systems, as well as reselling products from other manufacturers \cite{vetek}. 

% Vetek + IoT
Vetek aims to improve their services, and as such are interested in the possible use cases of IoT technology, and ultimately see how that can be applied to their own products. With something as simple as an IoT-enabled scale, they can offer customers products that can be placed in remote areas without the need for constant checkups, enabling long term monitoring and making it easier to analyze the data. An example might be monitoring road salt depots, to enable smarter refill routes during winter time, or a fodder station to map the behaviors of local wildlife.

% Challenges
The largest challenges for this type of device lies in energy consumption and broadcast range. Low energy consumption is needed so that any maintainer does not need to make constant check-ups to switch batteries all the time. This poses a limitation on the type of scale that can be used, which in extension affects parameters such as scale accuracy and capacity. A wide broadcast range is needed so that the device is not limited to being close to a base station. This puts restraints on what type of communication protocols can be used, as traditional ones such as Wi-Fi and Bluetooth will not work in the aforementioned examples.

% Available tech
With the advent of IoT, the 3:rd Generation Partnership Program, a standardization organization for telecom (3GPP) has developed new wireless communication protocols intended to be used by these devices \cite{3gpp}. One of these, the Narrowband-IoT (NB-IoT) protocol is particularly suitable for the challenges mentioned above, as its focus lies (among else) in wide area coverage and long battery life. A microcontroller (a small computer) is needed to handle the data polled from the scale, as well as sending it via some wireless communication protocol. The microcontroller chosen for this project is a FiPy, as it has the capability to handle multiple wireless technologies, one of them being NB-IoT. Some other essential pieces of hardware needed is some form of power supply, as well as an Analog-to-Digital Converter (ADC) that connects the microcontroller and the scale.

% What I did
In this paper, an attempt is made to implement a NB-IoT enabled load cell (the term load cell is interchangeable with scale for all intents and purposes). A functioning connection between the load cell and microcontroller could be established near the end of the project, though no real data transmission of this functional data was made due to time constraints. Instead, some fictional data was sent from the microcontroller itself to test its transmission capabilities.

% Road map to other chapters
%Background
In chapter 2, some general background is discussed, as well as some related work that pertains to the implementation challenges in IoT regarding energy efficacy and data error detection.
%Requirements
In order to emulate some behavior needed in the context of a real-world application, a few requirements have been placed on the device in the way that it handles the polling and transmission of data. The desired behavior is of an all-purpose IoT scale, with the requirements being set by Vetek.  These requirements will be presented in chapter 3.
%Methodology
The hardware and software implementation of the device will be presented in chapter 4.
% Results & Discussion
The outcome of the implementation will be presented in chapter 5 and
%Conclusions & future work
conclusions as well as a discussion of possible future work are brought up in the final chapter.
