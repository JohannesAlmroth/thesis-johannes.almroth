% Should be ~50% of the whole thesis
\iffalse
\begin{itemize}
	\item Don’t make the reader do all the work
	\item Have a hypothesis, test them, state result clearly
	\item Two lists are not a comparison
	\item Be the first to criticize your own work
\end{itemize}
\fi

With the help of the Otii battery toolbox, a functioning load cell-to-microcontroller setup was devised, though no end-to-end data transfer was made with the real data. Separate tests with the FiPy showed capabilities to send data via WiFi in conjunction with the PyCom platform, PyBytes. %% TODO: Do these tests
The device also fulfilled the software behavior requirements imposed to emulate real world situations and usage.

\section{Discussion}
%Interpretations: what do the results mean?
Despite hardships and complications when implementing the hardware, the results suggest that building and programming an NB-IoT enabled device connected to a load cell is possible. The necessary hardware components are neither too unavailable nor incompatible enough to hinder, at the very least, a small scale implementation. 

%Implications: why do the results matter?
These results show a small starting post of implementing a IoT device using a load cell as its sensor. With the rise of 5G and IoT platform services there exists a real commercial interest to enable more and more actors to innovate and create products for consumers and companies alike. This project can serve as a starting point for what basic needs and behaviors that need to be considered when planning and designing such a device.	

%Limitations: what can’t the results tell us?
%% Mass production
%% Variance for customer
%% Real life environments

%Recommendations: what practical actions or scientific studies should follow?
