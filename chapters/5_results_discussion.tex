% Should be ~50% of the whole thesis
\iffalse
\begin{itemize}
	\item Don’t make the reader do all the work
	\item Have a hypothesis, test them, state result clearly
	\item Two lists are not a comparison
	\item Be the first to criticize your own work
\end{itemize}
\fi

With the help of the Otii battery toolbox, a functioning load cell-to-microcontroller setup was devised, though no end-to-end data transfer via network was made with the real data. Separate tests with the FiPy showed capabilities to send data via WiFi in conjunction with the PyCom platform, PyBytes. %% TODO: Do these tests
The device also fulfilled the software behavior requirements imposed to emulate real world situations and usage, though it's worth repeating that these requirements were based on guesswork and assumption, rather than tests and research.

\section{Discussion}
%Interpretations: what do the results mean?
Despite hardships and complications when implementing the hardware, the results suggest that building and programming an NB-IoT enabled device connected to a load cell is possible with fairly simple consumer available hardware devices. It is a reasonable assumption that the hardware hindrances encountered in this paper can be bypassed with adequate planning and experience in assembling electric hardware.

%Implications: why do the results matter?
These results show a small starting post of implementing a IoT device using a load cell as its sensor. With the rise of 5G and IoT platform services there exists a real commercial interest to enable more and more actors to innovate and create products for consumers and companies alike. This project can serve as a starting point for what basic needs and behaviors that need to be considered when planning and designing such a device.	

\subsection{Limitations}
%Limitations: what can’t the results tell us?
%% Mass production
This project was done on the smallest scale possible, and extensive research in multiple different areas need to be conducted to even get an IoT enabled scale close to being produced for functional use, commercial or private.
%% Variance for customer
Furthermore, these results can't account for wether the software implementation was close to emulating the desired behavior of a device from a practical standpoint, since no potential end-users were involved in the requirement specification.
%% Real life environments
Since the testing was only done indoors in a clean and controlled environment, unknown variables present in the real world could very well produce a multitude of challenges that change the way that the device works. Another interesting angle is how different locations would interact with the connection to the cellular network it relies on for communication. The NB-IoT technology makes huge promises, but at the end of the day it is up to the local telecom company to fulfill the underlying conditions that make those claims possible, which would be Telia in our case.
% Different operators
Expanding on this, since the NB-IoT technology is fairly new in a lot of countries, there is bound to be extremely different implementation experiences from region to region depending on the network provider. In fact, NB-IoT devices could be rendered obsolete in entire regions depending on the network.
